% Created 2015-04-17 Fr 01:11
\documentclass[11pt]{article}
\usepackage[utf8]{inputenc}
\usepackage[T1]{fontenc}
\usepackage{fixltx2e}
\usepackage{graphicx}
\usepackage{longtable}
\usepackage{float}
\usepackage{wrapfig}
\usepackage{rotating}
\usepackage[normalem]{ulem}
\usepackage{amsmath}
\usepackage{textcomp}
\usepackage{marvosym}
\usepackage{wasysym}
\usepackage{amssymb}
\usepackage{hyperref}
\tolerance=1000
\date{\today}
\title{test\_konventionen}
\hypersetup{
  pdfkeywords={},
  pdfsubject={},
  pdfcreator={Emacs 24.4.1 (Org mode 8.2.10)}}
\begin{document}

\maketitle
\tableofcontents


\section{Test Konventionen}
\label{sec-1}

\subsection{Konvention}
\label{sec-1-1}
Tests werden nach folgendem Schema benannt:
\begin{verbatim}
Arbeitseinheit_Eingabe_ErwartetesErgebnis
\end{verbatim}
(Englisch: "UnitOfWork\_StateUnderTest\_ExpectedBehavior")

\begin{itemize}
\item Eine Arbeitseinheit ist die Klasse, Methode bzw. Funktionalität die getestet wird. \\
    Ist am getesteten Ablauf nicht nur eine Methode beteiligt wird als \\
    Arbeitseinheit die erste public Methode gewählt, die aufgerufen wird. ("Einstiegsmethode")
\item Die Eingabe sind die Eingabeparameter für den Test.
\item Das Erwartete Ergebnis ist das, was durch die Assertion(s) getestet wird.
\end{itemize}

\subsection{Beispiele}
\label{sec-1-2}

\begin{verbatim}
public void Sum_NegativeNumberAs1stParam_ExceptionThrown()

public void Sum_NegativeNumberAs2ndParam_ExceptionThrown ()

public void Sum_simpleValues_Calculated ()

public void Parse_OnEmptyString_ExceptionThrown()

public void Parse_SingleToken_ReturnsEqualToeknValue ()
\end{verbatim}

\begin{verbatim}
IsLoginOK_UserDoesNotExist_ReturnsFalse

AddUser_ValidUserDetails_UserCanBeLoggedIn

IsLoginOK_LoginFails_CallsLogger
\end{verbatim}

\subsection{Referenzen und weiterführende Texte}
\label{sec-1-3}
\begin{itemize}
\item \url{http://osherove.com/blog/2005/4/3/naming-standards-for-unit-tests.html}
\item \url{http://osherove.com/blog/2012/5/15/test-naming-conventions-with-unit-of-work.html}
\item \url{http://osherove.com/blog/2012/5/15/what-does-the-unit-in-unit-test-mean.html}
\item \url{http://artofunittesting.com/definition-of-a-unit-test/}
\end{itemize}
% Emacs 24.4.1 (Org mode 8.2.10)
\end{document}
